%%%%%%%%%%%%%%%%%%%%%%%%%%%%%%%%%%%%%%%%%%%%%%%%%%%%%%%%%%%%%%%%%%%%%%%%%%%%
\documentclass[12pt,a4paper]{beamer}           % pdfTeX の場合
\usetheme{Warsaw}
\usefonttheme{professionalfonts} % 数式のフォントをかっこよくする
\hypersetup{unicode}% しおりの和文出力に必要
\setbeamertemplate{headline}{
}
\setbeamertemplate{footline}{
	\leavevmode%
	\hbox{\begin{beamercolorbox}[wd=.5\paperwidth,ht=2.5ex,dp=1.125ex,leftskip=.3cm plus1fill,rightskip=.3cm]{author in head/foot}%
			\usebeamerfont{author in head/foot}\insertshortauthor
		\end{beamercolorbox}%
		\begin{beamercolorbox}[wd=.5\paperwidth,ht=2.5ex,dp=1.125ex,leftskip=.3cm,rightskip=.3cm plus1fil]{title in head/foot}%
			\usebeamerfont{title in head/foot}\insertshorttitle
			\hspace{\fill} \insertframenumber ~/ \inserttotalframenumber
	\end{beamercolorbox}}%
	\vskip0pt%
}
% 付録をページ数にカウントしない NPC
% appendixの後に,\backupbeginと書き,\end{document}の手前に,\backupendと書けばよい.
\newcommand{\backupbegin}{
	\newcounter{framenumberappendix}
	\setcounter{framenumberappendix}{\value{framenumber}}
}
\newcommand{\backupend}{
	\addtocounter{framenumberappendix}{-\value{framenumber}}
	\addtocounter{framenumber}{\value{framenumberappendix}}
}
% NPC
\usepackage[whole]{bxcjkjatype}% whole 指定 日本語使える

\usepackage{algorithm}
\usepackage{algorithmic}
%\usepackage{theorem}
\usepackage{amsthm}
\usepackage{amssymb}
\usepackage{amsmath}
\usepackage{bm}
\usepackage{bbm}
\usepackage{mathtools}
%\usepackage[dvipdfmx]{graphicx}
\usepackage{graphicx}
\usepackage{overpic}
\usepackage[absolute, overlay]{textpos}
\TPoptions{absolute=false}
%\setbeamertemplate{background}[grid][step=30pt]
\usepackage{tikz}

\newcommand{\bhrule}{{\color{blue} \hrule}}
\newcommand{\rect}[5]{
	\TPoptions{absolute=true}
	\begin{textblock*}{\textwidth}(#1pt,#2pt)
		\begin{tikzpicture}
			\draw[#5,thick] (0pt,0pt) rectangle (#3pt,#4pt);
		\end{tikzpicture}
	\end{textblock*}
	\TPoptions{absolute=false}
}

\definecolor{nico}{rgb}{1, 0.90, 0.96}
\definecolor{orange}{rgb}{0.95, 0.3, 0.0}
\definecolor{ash}{rgb}{0.8,0.8,0.8}
\renewcommand{\alert}[1]{{\color{orange} {#1} }}
\newcommand{\head}[1]{ \hspace{-10pt}\structure{#1} \\ \vspace{5pt} }
\newcommand{\subhead}[1]{ \vspace{5pt} \hspace{-10pt}\structure{#1} \\ \vspace{5pt} }


\newcommand{\p}{{\color{red}+1}}
\newcommand{\n}{{\color{blue}-1}}

%%%%%%%%%%%%%%%%%%%%%%%%%%%%%%%%%%%%%%%%%%%%%%%%%%%%%%%%%%%%%%%%%%%%%%%%%%%

\begin{document}

% 和文タイトル
\title{}
\author{横山侑政}

% \maketitle

%%%%%%%%%%%%%%%%%%%%%%%%%%%%%%%%%%%%%%

\section{準備}
%------------------------------
\begin{frame}{全部分グラフの決定木勾配ブースティング}
	\head{グラフの分類回帰問題を解く}
	\vspace{-10pt}
\begin{minipage}[t]{0.24\hsize}
	\centering
	\includegraphics[width=50pt]{graph/g03b.png} \\
	\n
\end{minipage}
\begin{minipage}[t]{0.24\hsize}
	\centering
	\includegraphics[width=50pt]{graph/g01r.png} \\
	\p
\end{minipage}
\begin{minipage}[t]{0.24\hsize}
	\centering
	\includegraphics[width=50pt]{graph/g06r.png} \\
	\p
\end{minipage}
\begin{minipage}[t]{0.24\hsize}
	\centering
	\includegraphics[width=50pt]{graph/g07b.png} \\
	\n
\end{minipage}
\\
\vspace{5pt}

	\subhead{特徴量は部分グラフの有無}
	\begin{textblock*}{\textwidth}(180pt,90pt)
	\alert{部分グラフの数が膨大}
\end{textblock*}

\vspace{-10pt}
\begin{table}
	\hspace{-30pt}
	\begin{tabular}{cc|ccccc}
		~ & ~ &
		\shortstack{ $x_1$ \\ \includegraphics[width=30pt]{subgraph/kw.png} } &
		\shortstack{ $x_2$ \\ \includegraphics[width=40pt]{subgraph/kwg.png} } &
		\shortstack{ $x_3$ \\ \includegraphics[width=45pt]{subgraph/kwgy.png} } &
		\shortstack{ $x_4$ \\ \includegraphics[width=30pt]{subgraph/wg.png} } &
		\raisebox{5pt}{$\cdots$} \\
		\hline
		\parbox[c][30pt][c]{0pt} {$G_1$} & \raisebox{-3pt}{\includegraphics[width=40pt]{graph/g03b.png}} & 1 & 0 & 0 & 0  & $\cdots$ \\
		\parbox[c][30pt][c]{0pt} {$G_2$} & \raisebox{-3pt}{\includegraphics[width=40pt]{graph/g01r.png}} & 0 & 0 & 0 & 1 & $\cdots$ \\
		% \parbox[c][30pt][c]{0pt} {$G_3$} & \raisebox{-18pt}{\includegraphics[width=50pt]{graph/g06r.png}} & 1 & 1 & 0 & 1 & \\
		% \parbox[c][30pt][c]{0pt} {$G_4$} & \raisebox{-10pt}{\includegraphics[width=50pt]{graph/g07b.png}} & 1 & 1 & 1 & 1 & \\
	\end{tabular}
\end{table}

	\subhead{特徴ベクトルを陽には計算せず学習する}
\end{frame}



%------------------------------
% \begin{frame}{~}
% 	\begin{center}
% 		\textcolor[rgb]{0.0, 0.0, 1.0}{T}
% 		\textcolor[rgb]{0.0, 0.0, 0.8}{h}
% 		\textcolor[rgb]{0.0, 0.0, 0.6}{a}
% 		\textcolor[rgb]{0.0, 0.0, 0.4}{n}
% 		\textcolor[rgb]{0.0, 0.0, 0.2}{k}
% 		\textcolor[rgb]{0.0, 0.0, 0.0}{s}
% 	\end{center}
% \end{frame}



%%%%%%%%%%%%%%%%%%%%%%%%%%%%%%%%%%%%%%

\end{document}

